\begin{acknowledgements}

Firstly, I would like to thank my supervisor, Professor Ricky Yu-Kwong Kwok, for his kindly and thoughtful support during my PhD study. The works presented in this thesis, to some degree, are out of my pure personal interest in searching for a crossing point of music and computational technology. They can not even be ever started, if I was not given enough trust and encouragement. Throughout these 4 years time, Professor Kwok and I have a lot of constructive and meaningful conversations about the dual mission of career and life, which I believe are beneficial in the rest of my life.

I also thank Professor Francis Chi-moon Lau, for his appreciation of my early works. In the first two years, we communicate a lot in terms of music expression interfaces. I am deeply impressed by his enthusiasm and sophisticated knowledge in both computer and music. His comments make a lot of difference in this thesis.

Thank Doctor Xiao Hu, one of the most influential researchers in music information retrieval, for her very high research standard, which significantly improve the quality of this thesis.

Thank Professor Andrew Horner, one of the most renowned computer music researchers, for his insightful reviews and comments on this thesis, which really boost it to another level.

I would like to express my deepest gratitude to Qing, for her everlasting supportive and positive point of view towards everything I have done. I truly enjoy all those fruitful discussions about music industry, songwriting, hip-hop and jazz. Her passion about music technologies always makes me believe that what I have been doing is worthwhile.

I am indeed grateful to be living in the Graduate House, where I gradually improve my spoken English, perfect my accompaniment skills, and learn jazz. My life would have been dry and bored without my house friends. Thank my songwriting partner, the real piano accompanist Jonathan de la Cruz. He plays wonderful gigs, sing-along sessions, and always allows me to watch and learn from his sophisticated techniques. Thank Yi Eun, the real musician, for his sincere appreciation of my music. Thank Kelvin, the real singer, for his one of the best vocals in the world. Thank Tianyin, for her nice small gifts, Faichuns, and cookings. Thank JT, for training me to be a better ``drinker''. There are a lot more people I want to thank: Billy and Aimie, Ben and Haidi, Chae Yin, Maggie, Wenpin, Haoyuan, Dinghua, Tewei, Lili, Shengda, Pingyu, Xiaowan, Dandan, Jay Lu, the music group, the card game group, the fitness group, the dancing group, the list can go on and on... I am just so reluctant to leave and start a new life.

My sincere thanks also go to my labmates in Chow Yei Ching Building 101. Here I met Ho-Cheung, with whom I share a lot of life stories. I met Bony, Dominic, and Sam. They are always willing to help, welcoming to talk, and passionate to share all kinds of ideas. I would also like to thank Xing. He is one of the best colleagues I have in EEE with signal processing and machine learning background. He makes me feel that I am not alone. Finally, I am always grateful to Xiangyu during the teaching assistant time. He is humble, hardworking, strong-minded and he never gives up.

I also thank Yuheng, my best friend in Guangzhou, to always have my back when I am in trouble.

Lastly, I take this opportunity to thank my wife, my parents and my close relatives. They are always there when I am in need and they always understand me. I am greatly indebted to them, for not being able to be around, and not always doing what they expected. For everything that I have done wrong, I am deeply sorry.

Thank you for all the memories in these years:
\begin{quote}
\centering
\textit{If I should live forever}

\textit{And all my dreams come true}

\textit{My memories of love will be of you}

\textit{--- John Denver}
\end{quote}

\end{acknowledgements}




