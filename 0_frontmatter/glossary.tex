% this file is called up by thesis.tex
% content in this file will be fed into the main document

% Glossary entries are defined with the command \nomenclature{1}{2}
% 1 = Entry name, e.g. abbreviation; 2 = Explanation
% You can place all explanations in this separate file or declare them in the middle of the text. Either way they will be collected in the glossary.

% required to print nomenclature name to page header
\markboth{\MakeUppercase{\nomname}}{\MakeUppercase{\nomname}}

\nomenclature{ACE}{automatic chord estimation}
\nomenclature{LV}{large vocabulary}
\nomenclature{MIR}{music information retrieval}
\nomenclature{MIREX}{music information retrieval exchange}
\nomenclature{ASR}{automatic speech recognition}
\nomenclature{DFT}{discrete Fourier transform}
\nomenclature{STFT}{short-time Fourier transform}
\nomenclature{PCP}{pitch class profile}
\nomenclature{CQT}{constant-Q transform}
\nomenclature{HPCP}{harmonic pitch class profile}
\nomenclature{NNLS}{non-negative least square}
\nomenclature{PCA}{principle component analysis}
\nomenclature{ICA}{independent component analysis}
\nomenclature{MLP}{multi-layer perceptron}
\nomenclature{DBN}{deep belief network}
\nomenclature{RNN}{recurrent neural network}
\nomenclature{BRNN}{bidirectional recurrent neural network}
\nomenclature{DNN}{deep neural network}
\nomenclature{CNN}{convolutional neural network}
\nomenclature{LSTM}{long short-term memory}
\nomenclature{BLSTM}{bidirectional long short-term memory}
\nomenclature{RBM}{restricted Boltzmann machine}
\nomenclature{SGD}{stochastic gradient descent}
\nomenclature{CD}{contrastive divergence}
\nomenclature{PCD}{persistent contrastive divergence}
\nomenclature{HMM}{hidden Markov model}
\nomenclature{CRF}{conditional random field}
\nomenclature{DYBM}{dynamic Bayesian network}
\nomenclature{MLN}{Markov logic network}
\nomenclature{SVM}{support vector machine}
\nomenclature{GMM}{Gaussian mixture model}
\nomenclature{RCO}{relative correct overlap}
\nomenclature{CSR}{chord symbol recall}
\nomenclature{WCSR}{weighted chord symbol recall}
\nomenclature{SQ}{segmentation quality}
\nomenclature{ACQA}{Average Chord Quality Accuracy}

% ----------------------- contents from here ------------------------

% chemicals
%\nomenclature{DAPI}{4',6-diamidino-2-phenylindole; a fluorescent stain that binds strongly to DNA and serves to marks the nucleus in fluorescence microscopy} 
%\nomenclature{DEPC}{diethyl-pyro-carbonate; used to remove RNA-degrading enzymes (RNAases) from water and laboratory utensils}
%\nomenclature{DMSO}{dimethyl sulfoxide; organic solvent, readily passes through skin, cryoprotectant in cell culture}
%\nomenclature{EDTA}{Ethylene-diamine-tetraacetic acid; a chelating (two-pronged) molecule used to sequester most divalent (or trivalent) metal ions, such as calcium (Ca$^{2+}$) and magnesium (Mg$^{2+}$), copper (Cu$^{2+}$), or iron (Fe$^{2+}$ / Fe$^{3+}$)}



