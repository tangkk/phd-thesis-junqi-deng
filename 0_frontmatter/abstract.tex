
% Thesis Abstract -----------------------------------------------------


%\begin{abstractslong}    %uncommenting this line, gives a different abstract heading
\begin{abstracts}        %this creates the heading for the abstract page
This thesis presents two novel ways of solving the large vocabulary (LV) automatic chord estimation (ACE) problem with deep learning technologies, and tries to apply these solutions within an automatic jazz improvisation context.

Being well aware of the annotation subjectivity issue, this thesis asks for the necessity of LV with a joint argument of machine musicianship and Turing test. Built upon this premise, it proposes two system frameworks that lead to practical LVACE implementations.

The first framework considers the separation of segmentation and classification during ACE, while all previous literatures combine the two processes in one single pass. Three types of deep learning models are proposed for classification. Under large vocabulary evaluation, two of them are over-fitting chords with large populations, while the other model shows great potential in balanced performance. This framework has already proved its successfulness in the music information retrieval evaluation exchange (MIREX) 2016 ACE tasks.

The second framework considers an end-to-end neural network approach, and employs an ``even chance'' training technique to make up for the lack of long-tail chord examples. As revealed in the evaluation, the main drawback of this approach is the low segmentation quality, which inevitably lowers the upper-bound of chord recognition accuracy. Nevertheless, it demonstrates a preliminary successfulness of a fully end-to-end approach without any post-processing stages, and also proves the even chance training scheme to be effective for long-tail classification problem.

Finally, as a preliminary study, a jazz chord estimation system is built from the first framework. Then a chord-scale estimation system is built with a template-based scale tracking algorithm. Some semi-automatic or fully automatic jazz improvisation demos are created out of the combination of the chord-scale tracker, two jazz improvisation musical interfaces, and a Markov model based note generator.


\end{abstracts}
%\end{abstractlongs}


% ---------------------------------------------------------------------- 
