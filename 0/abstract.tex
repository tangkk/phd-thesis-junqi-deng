
This thesis presents two novel ways of solving the large vocabulary automatic chord estimation problem with deep learning technologies, and tries to apply these solutions within an automatic jazz improvisation context. Being well aware of the annotation subjectivity issue, this thesis attests the necessity of large vocabulary with a joint argument of machine musicianship and the Turing test. Built upon this premise, it proposes two deep learning related system frameworks that lead to potential practical solutions to large vocabulary automatic chord estimation.

The first framework considers separating chord segmentation and classification into two tasks, which is unlike all previous approaches that combine the two in one single pass. Three deep learning models are proposed for classification. Under large vocabulary evaluation, two of the models mainly over-fit major and minor triads, while the third model shows great potential in balanced performances across different classes. This framework has proved its ability over large vocabulary evaluation in the automatic chord estimation task of music information retrieval evaluation exchange 2016.

The second framework considers an end-to-end neural network approach. It employs an ``even chance'' training technique to make up for the lack of uncommon chords' exposure. The main drawback of this approach is the low segmentation quality, which inevitably lowers the upper-bound of the overall chord symbol recall. Nevertheless, it demonstrates preliminary success of a fully end-to-end approach without post-processing, and also proves the even chance training scheme to be effective for the skewed classes classification problem.

Finally, a preliminary study has been conducted for automatic jazz chord estimation. Upon this study, a chord-scale estimation system is built with a template-based scale tracking algorithm. Some semi-automatic or fully automatic jazz improvisation demos are created out of the combinations of the chord-scale tracker, two jazz improvisation musical interfaces, and a Markov model based note generator.
