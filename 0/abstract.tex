Being well aware of the chord annotation subjectivity issue, this thesis attests the necessity of large vocabulary with a joint argument of machine musicianship and the Turing test. Built upon this premise, it proposes two deep learning based system frameworks that lead to potential practical solutions to large vocabulary automatic chord estimation.

The first framework separates chord segmentation and classification into two tasks, which is unlike all previous approaches that combine them in one single pass. Several deep learning models are implemented and tested. Under the large vocabulary evaluation, the recurrent neural network model shows great potential in balanced performances across different chords. This framework has shown its advantages over large vocabulary evaluation in the automatic chord estimation task of music information retrieval evaluation exchange 2016.

The second framework incorporates a skewed class distribution sensitive approach. It employs an ``even chance'' scheme to boost the uncommon chords' exposure when training a recurrent neural network sequence decoder. The main drawback of this approach is the low segmentation quality. Nevertheless, it demonstrates the even chance training scheme to be effective for the large vocabulary automatic chord estimation.

Finally, a preliminary study has been conducted for automatic jazz chord estimation. Upon this study, a chord-scale estimation system is built and some semi-automatic or fully automatic jazz improvisation demos are created.
